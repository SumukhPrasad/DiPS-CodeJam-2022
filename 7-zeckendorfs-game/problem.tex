\documentclass[12pt]{report}
\usepackage[a4paper, total={7.3in, 9.7in}]{geometry}
\usepackage{amsmath}
\usepackage{upquote}
\usepackage{listings}
\usepackage{xcolor}
\usepackage{titlesec}
\usepackage{amssymb}

\definecolor{backgroundcolor}{rgb}{1, 1, 1}
\definecolor{commentstyle}{rgb}{0.365, 0.422, 0.475}
\definecolor{keywordstyle}{rgb}{0.6, 0.14, 0.576}
\definecolor{numberstyle}{rgb}{0.5, 0.5, 0.5}
\definecolor{stringstyle}{rgb}{0.77, 0.1, 0.08}

\lstdefinestyle{xcodecolor}{
    backgroundcolor=\color{backgroundcolor},   
    commentstyle=\color{commentstyle},
    keywordstyle=\color{keywordstyle},
    numberstyle=\scriptsize\color{numberstyle},
    stringstyle=\color{stringstyle},
    basicstyle=\ttfamily\footnotesize,
    breakatwhitespace=false,         
    breaklines=true,                 
    captionpos=b,                    
    keepspaces=true,                   
    numbersep=5pt,                  
    showspaces=false,                
    showstringspaces=false,
    showtabs=false,                  
    tabsize=2
}

\lstset{style=xcodecolor}

\usepackage[T1]{fontenc}
\usepackage{cascadia-code}

% Raised Rule Command:
%  Arg 1 (Optional) - How high to raise the rule
%  Arg 2            - Thickness of the rule
\newcommand{\raisedrule}[2][0em]{\leaders\hbox{\rule[#1]{1pt}{#2}}\hfill}

\setlength{\parindent}{0pt}
\titleformat{\section}
  {\normalfont\Large\bfseries}{\thesection}{1em}{}[{\titlerule[0.8pt]}]
\begin{document}

	{\Large
	\textbf{Zeckendorf's Game}}
	
	\vspace{0.4cm}
	DiPS CodeJam 22\raisedrule[0.25em]{1pt}
	\\
	% document

  \section*{Prompt}
  In a game of \textit{Zeckendorf}, your task is to find the shortest representation of a given integer as a sum of Fibonacci numbers. For example, the \textit{Zeckendorf} representation of 10 is $10 = 2 + 8$. Numbers \textbf{cannot} be repeated.\\
  Pranav and Prithvi are playing a game of Zeckendorf. Can you help them find the answers as fast as possible?
  \subsection*{Input Format}
  The first and only line of input will contain an integer $n$.
  \subsection*{Output Format}
  The first and only line of your output must contain a space-separated list of the \textit{Zeckendorf} representation of $n$, sorted in ascending order.
  \subsection*{Constraints}
  $ 1 \le n \le 10^5 $
  \subsection*{Sample Input/Output}
  \begin{tabular}{ |l|l| } 
    \hline
    \textbf{Input} & \textbf{Output} \\
    {\lstinputlisting{./testCases/input/input00.txt}} & {\lstinputlisting{./testCases/output/output00.txt}} \\ % use \lstinputlisting{./testCases/<in/out>put/<in/out>put00.txt}
    \hline
   \end{tabular}


  \section*{Solution}
  Let's take the sample input (93743) as $n$. To find the \textit{Zeckendorf} representation, we\\
  \begin{itemize}
    \item  First, find the greatest Fibonacci Number smaller than or equal to $n$.
    \item Append the fibonacci number we found to an array.
    \item Reduce $n$ by $f$ ($n=n-f$).
    \item Repeat these steps while $n>0$.
    \item We now have an array of the \textit{Zeckendorf} representation of $n$. To obtain the result, we sort the array in ascending order, and then print it.
  \end{itemize}
	\section*{Sample Program}
	\lstinputlisting[language=Python]{sampleSolution.py}
	

\end{document}
% https://codegolf.stackexchange.com
% https://codegolf.stackexchange.com/questions/250745/cryptic-multiplications
% https://codegolf.stackexchange.com/questions/49943/maximum-concatenated-product?rq=1
% https://codegolf.stackexchange.com/questions/194869/do-you-make-me-up
% https://en.wikipedia.org/wiki/Zeckendorf%27s_theorem
% https://codereview.stackexchange.com/questions/36655/count-possible-paths-through-a-maze
% Game of Life - next iter find.